\documentclass[a4paper, 11pt]{article}
\usepackage[utf8]{inputenc} % Change according your file encoding
\usepackage{graphicx}
\usepackage{url}
\usepackage[margin=1in]{geometry}
\usepackage{caption}
\usepackage{subcaption}
\usepackage{float}

\usepackage[catalan,english]{babel}


%% Estil de Paràgraf
\setlength{\parskip}{4mm}
\setlength{\parindent}{0mm}

%% Estil lletra
\renewcommand{\familydefault}{\sfdefault}


%% Estil de Paràgraf
\setlength{\parskip}{4mm}
\setlength{\parindent}{0mm}

%opening
\title{Seminar Report: Mutty}
\author{Martín Garcia \and Ferran Arau}
\date{\today{}}

\begin{document}

\maketitle

\section{Introducció}

% Introduce in a couple of sentences the seminar and the main topic related to
% distributed systems it covers.

El seminari té com objectiu implementar un sistema d'exclusió mutua entre diferents nodes d'un sistema distribuït. 

Cada node disposa d'una regió crítica associada. En aquest cas la regió crítica és un codi artificial que ''dorm'' durant un temps quan és executat, s'anomena \textit{worker}. No pot haver més d'un node executant el seu \textit{worker} simultàniament. 

Donat l'escenàri exposat cal implementar en erlang el sistema de comunicacions entre els processos per tal de garantir l'exclusió mutua, així com els possibles deadlocks. 

La tècnica utilitzada per dur a terme la implementació és l'exposada pel document \textit{An optimal Algorithm for mutual Exclusion in Computer Networks} publicat per en Glen Ricart i l’Ashok K. Agrawala.

\section{Feina de laboratori}

A continuació es presenten les tres implementacions del sistema de locks. Es
mostren les diferències entre el codi original i el de producció mitjançant les
eines de ''history'' que proporciona el repositori utilitzat. 

Es pot consultar el codi d'aquesta sessió a
\url{https://github.com/magarcia/SDX/tree/master/S2}

La primera implementació és donada per l'assignatura i no ha estat necessari modificar-la. No obstant, s'han fet les proves recomanades. S'ha pogut apreciar els diversos comportaments del sistema modificant els paràmetres de \texttt{Sleep} i \texttt{Work}. \\ 
A continuació es mostren les estadístiques generades a partir de les proves.

La segona implementació consisteix en solucionar els possibles deadlocks. Per fer-ho s'ha utilitzat un sistema de prioritats a partir d'identificadors lògics de node. El node més petit és el que té prioritat, alhora d'entrar a la regió crítica, en cas d'empat. 



\section{Preguntes directes}

% Try to answer all the open questions in the documentation. When possible, do
% experiments to support your answers.

\section{Opinió personal}


\end{document}

